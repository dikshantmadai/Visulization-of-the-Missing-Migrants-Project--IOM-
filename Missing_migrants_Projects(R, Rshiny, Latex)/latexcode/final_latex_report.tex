\documentclass{article}

\usepackage{Sweave}
\usepackage{graphicx}
\usepackage{hyperref}
\usepackage{datetime}

\title{\Huge CMP5352: Data Visualisation Report}
\author{\Large Dikshant Madai (23140738)}

\begin{document}

\maketitle
\begin{center}
    \textbf{Subject:} {CMP5352: Data Visualisation}
\end{center}
\begin{center}
    \textbf{Word count:}{11 }  % Placeholder for word count
\end{center}



\input{final_latex_report-concordance} 
\maketitle
\newpage
\tableofcontents
\newpage

\begin{abstract}
This research examines data from the International Organization for Migration's Missing Migrants Project, presenting a graphic portrayal of the global tragedy of migrant fatalities and disappearances between 2014 and 2023. The data suggests that the Mediterranean Sea remains the most dangerous route, with a consistently high number of missing migrants. 2016 saw the largest number of missing migrants and deaths in a single year. Despite this sad reality, the number of survivors has fluctuated over time, with the highest number recorded in 2016.  A considerable gender disparity was seen across regions, with more males reporting missing than females.The Central Mediterranean route, particularly between Libya and Italy, has been highlighted as the most perilous area for migrants, with the highest number of missing people. Finally, the research ranks Afghanistan as the nation of origin with the highest number of migrant deaths, followed by Ethiopia, Mali, Algeria, and Côte d'Ivoire. This report aims to expose the global epidemic of missing migrants while also informing initiatives to improve safety and solve the challenges that individuals confront on their journeys.
\end{abstract}

\newpage

\section{Introduction}
This dataset contains a detailed record of missing migrants and their tragic journeys to overseas destinations, gathered by the Missing Migrants Project, an initiative launched by the overseas Organization for Migration (IOM) in 2014. The collection tracks deaths and disappearances, shedding insight on the difficulties migrants confront on their journeys. Please keep in mind that because of the complexity of data collecting, the figures reported are most likely an undercount. The dataset is a monument to the individuals who died, as well as the families and communities affected by their absence.
This report uses data visualization to analyze patterns in migrant deaths and disappearances recorded by the International Organization for Migration's Missing Migrants Project. It aims to raise awareness of this global issue and inform efforts to improve migrant safety. The report explores trends by region, location, migration route, and country of origin. By visualizing the data, we can identify areas with the highest incidence of missing migrants and understand how these patterns have evolved over time.
Missing Migrants Project collects data through a variety of sources, including:The Missing 

\begin{itemize}
    \item  Migrants Project collects data through a variety of sources
    \item  Media Reports
    \item  Witness Accounts
    \item  Government and NGO Reports
    \item Community Reports
\end{itemize}


\section{Motivation and Objectives}

The Missing Migrants Project is a crucial resource for understanding the dangers faced by migrants worldwide. This report utilizes the project's data to answer the following questions:
\begin{itemize}
    \item Which regions have the highest number of migrant deaths and disappearances, and how do these numbers vary over time?
    \item What insights can we gain from analyzing the total number of missing migrants each year? which year is the highest number of the missing migrants and death.

    \item Trend in the Number of Survivors Over the Years. which year is the more Survivors?
    \item What is the gender ratio of missing migrants across different regions?
    \item Which is the top  impact location of death on number of missing migrants?
    \item Which migration route has the highest incidence of missing persons, and how does it compare to others?
    \item which is the most countries of origin by number of migrant deaths  ?
\end{itemize}

The data provides valuable insights  to guide policymakers and humanitarian organizations in their efforts to address migrant safety.

\section{Data Description}
It has the record of missing migrants from 2014 to 2023
Here is the information about the dataset :

\begin{itemize}
    \item \textbf{Incident Type}: Type of migration incident
    \item \textbf{Incident Year}: Year when the incident happened
    \item \textbf{Reported Month}: Month when the incident was reported.
    \item \textbf{Region of Origin}: Geographical region where the migrants originated.
    \item \textbf{Region of Incident:}: The region where the incident occurred
    \item \textbf{Country of Origin:}:Country from where the migrants came
    \item \textbf{Number of Dead}: Number of confirmed dead migrants
    \item \textbf{Minimum Estimated Number of Missing}: Minimum estimated number of missing migrants.
    \item \textbf{Total Number of Dead and Missing}: Total number of deceased and missing migrants
    \item \textbf{Number of Survivors}:The number of migrants that survived the incident.
    \item \textbf{Number of Females}: Number of female migrants involved.
    \item \textbf{Number of Males}: Number of male migrants involved.
    \item \textbf{Number of Children}: Number of children migrants involved.
    participated.
    \item \textbf{Cause of Death}: Cause of death among migrants.
    \item \textbf{Migration Route}: Route used by migrants during their journey (if available)
    \item \textbf{Location of Death:}: Description of the location where the incident occurred.
    \item \textbf{Information Source}: Source of information about the incident.
    \item \textbf{Coordinates}: Approximately where the incident occurred.
    \item \textbf{UNSD Geographical Grouping}: Geographic classification according to the United Nations Statistics Division
    
\end{itemize}


Data source: \url{https://www.kaggle.com/datasets/nelgiriyewithana/global-missing-migrants-dataset} (click here)


\newpage
\section{Visualization}
\begin{Schunk}
\begin{Sinput}
> # Libraries
> library(ggplot2)
> library(reshape2)
> library(dplyr)
> library(maps)
> library(tidyr)
> library(bbplot) 
> library(lubridate)
> library(readr)
\end{Sinput}
\end{Schunk}

After cleaning this data set i have loaded it.
\begin{Schunk}
\begin{Sinput}
> # Loading Dataset and information about it
> file_path <- "D:/Data science/Dikshant_Madai/Dikshant_Madai/clean_dataset/migrants_clean.csv"
> # Read the CSV file into a data frame
> df <- read.csv(file_path)
> # Dealing qith missing values
> # Get the number of rows and columns
> num_rows <- nrow(df)
> num_cols <- ncol(df)
> cat("Number of rows:", num_rows, "\n")
\end{Sinput}
\begin{Soutput}
Number of rows: 13020 
\end{Soutput}
\begin{Sinput}
> cat("Number of columns:", num_cols, "\n")
\end{Sinput}
\begin{Soutput}
Number of columns: 19 
\end{Soutput}
\begin{Sinput}
> # Calculate missing values for each column
> missing_values <- sapply(df, function(x) sum(is.na(x)))
> print(missing_values)
\end{Sinput}
\begin{Soutput}
                      Incident.Type                       Incident.Year 
                                  0                                   0 
                     Reported.Month                    Region.of.Origin 
                                  0                                   0 
                 Region.of.Incident                   Country.of.Origin 
                                  0                                   0 
                     Number.of.Dead Minimum.Estimated.Number.of.Missing 
                                  0                                   0 
   Total.Number.of.Dead.and.Missing                 Number.of.Survivors 
                                  0                                   0 
                  Number.of.Females                     Number.of.Males 
                                  0                                   0 
                 Number.of.Children                      Cause.of.Death 
                                  0                                   0 
                    Migration.Route                   Location.of.Death 
                                  0                                   0 
                 Information.Source                         Coordinates 
                                  0                                   0 
         UNSD.Geographical.Grouping 
                                  0 
\end{Soutput}
\end{Schunk}

\begin{Schunk}
\begin{Sinput}
> # bbc_style customize theme
> bbc_style <- function() {
+  font <- "Helvetica"
+ 
+  ggplot2::theme(
+    plot.title = ggplot2::element_text(
+      family = font, size = 15, face = "bold", color = "#222222", hjust = 0.5),
+    plot.subtitle = ggplot2::element_text(
+      family = font, size = 18, margin = ggplot2::margin(9, 0, 9, 0)),
+    plot.caption = ggplot2::element_blank(),
+ 
+    legend.position = "top",
+    legend.text.align = 0,
+    legend.background = ggplot2::element_blank(),
+    legend.title = ggplot2::element_blank(),
+    legend.key = ggplot2::element_blank(),
+    legend.text = ggplot2::element_text(
+      family = font, size = 10, color = "#222222", hjust = 0.70),
+ 
+    axis.title = ggplot2::element_blank(),
+    axis.text = ggplot2::element_text(
+      family = font, size = 11, color = "#222222"),
+    axis.text.x = ggplot2::element_text(margin = ggplot2::margin(5, b = 10)),
+    axis.ticks = ggplot2::element_blank(),
+    axis.line = ggplot2::element_blank(),
+ 
+    panel.grid.minor = ggplot2::element_blank(),
+    panel.grid.major.x = ggplot2::element_blank(),
+    panel.background = ggplot2::element_blank(),
+ 
+    strip.background = ggplot2::element_rect(fill = "white"),
+    strip.text = ggplot2::element_text(size = 20, hjust = 0)
+  )
+ }
\end{Sinput}
\end{Schunk}

\begin{Schunk}
\begin{Sinput}
> # Dealing with missing values
> # Get the number of rows and columns
> num_rows <- nrow(df)
> num_cols <- ncol(df)
> cat("Number of rows:", num_rows, "\n") 
\end{Sinput}
\begin{Soutput}
Number of rows: 13020 
\end{Soutput}
\begin{Sinput}
> cat("Number of columns:", num_cols, "\n") 
\end{Sinput}
\begin{Soutput}
Number of columns: 19 
\end{Soutput}
\end{Schunk}

\subsection {Heatmap of Correlation Matrix}
\begin{Schunk}
\begin{Sinput}
> # 1). Heatmap of Correlation Matrix
> 
> # Filter numeric columns
> numeric_df <- df %>% select_if(is.numeric)
> # Compute correlation matrix
> correlation_matrix <- cor(numeric_df, use = "complete.obs")
> # Melt the correlation matrix for plotting
> melted_correlation <- melt(correlation_matrix)
> heatmap_plot <- ggplot2::ggplot(melted_correlation, 
+                                 aes(Var1, Var2, fill = value)) +
+  ggplot2::geom_tile(color = "white") +
+  ggplot2::scale_fill_gradient2(low = "blue", high = "red", mid = "white",
+                       midpoint = 0, limit = c(-1, 1), space = "Lab",
+                       name = "Correlation") +
+  ggplot2::theme_minimal() +
+  ggplot2::theme(plot.title = ggplot2::element_text(
+    size = 19, face = "bold", color = "#222222"),
+        axis.text.x = ggplot2::element_text(
+          angle = 45, vjust = 1, size = 10, hjust = 1, face="bold"),
+        axis.text.y = ggplot2::element_text(
+          face="bold", size = 10),
+        panel.grid.major = ggplot2::element_blank(),
+        panel.grid.minor = ggplot2::element_blank(),
+        legend.justification = c(1, 0),
+        legend.direction = "vertical") +
+  ggplot2::labs(title = "Correlation Heatmap",
+       x = "",
+       y = "")
> # Print the plot
> print(heatmap_plot)
\end{Sinput}
\end{Schunk}
\includegraphics{final_latex_report-005}


The heatmap shows the correlation between different variables related to migration incidents.There is also a strong positive correlation between the number of dead and the total number of dead and missing. The heatmap shows that there is a strong correlation between the number of deaths and the number of missing people. This suggests that as the number of deaths increases, the number of missing people also tends to increase.

\subsection {Regions with highest number of migrant deaths and disappearances
over time}


The map shows the locations of migration incidents around the world. The color of the dots represents the number of incidents at each location. The red dots represent the highest number of incidents, while the blue dots represent the lowest number of incidents. The map shows that the majority of migration incidents occur in the Mediterranean Sea,  the South China Sea and more.


\subsection {Region with Highest  Number of Incident}
<<fig=TRUE,echo=TRUE,fig.width=8, fig.height=8,>>

# 3). Which regions have the highest number of incidents?


# Create a date column for easier grouping by year
df$date <- as.Date(paste0(df$`Incident Year`, "-01-01"))

# 1. Grouping by region to find total missing migrants
region_data <- df %>%
 group_by(`Region of Incident`) %>%
 summarise(total_missing = sum(`Total Number of Dead and Missing`,
                               na.rm = TRUE)) %>%
 arrange(desc(total_missing))

# Create the bar chart
ggplot2::ggplot(region_data, aes(x = reorder(`Region of Incident`,
                                             total_missing), 
                                 y = total_missing)) +
 ggplot2::geom_bar(stat = "identity", fill = "darkred") +
 ggplot2::coord_flip() + # Flip coordinates for better readability
 ggplot2::geom_text(aes(label = total_missing), hjust = -0.09, 
                    color = "black", size=2.5) + 
 ggplot2::labs(
   title = "Regions with the Highest Number of Missing Migrants",
   x = NULL, 
   y = "Total Missing Migrants"
 ) +
 bbc_style() + 
 ggplot2::theme(
   panel.grid.major = ggplot2::element_blank(), 
   panel.grid.minor = ggplot2::element_blank(), 
   axis.title.x = ggplot2::element_blank(), 
   axis.text.x = ggplot2::element_blank(),  
   axis.ticks.x = ggplot2::element_blank(),
   axis.text.y = ggplot2::element_text(margin = ggplot2::margin(r = -15)) 

 )


The graph depicts the number of missing migrants according on region. The Mediterranean region has the highest number of missing migrants, followed by North Africa and North America. The amount of missing migrants reduces as you go down the list, with Eastern Asia having the fewest. The graph indicates that missing migrants have a disproportionate impact in the Mediterranean region. This is most likely owing to the area's proximity to war zones and the large number of individuals attempting to cross the Mediterranean Sea to reach Europe. 


\subsection { Most Affected Regions}
\begin{Schunk}
\begin{Sinput}
> # Filtering for the top 5 most affected regions
> top_regions <- region_data %>%
+  top_n(5) %>%
+  pull(`Region of Incident`)